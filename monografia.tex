\documentclass[a4paper,12pt,openright,oneside]{book}

% Enables Portuguese Brasil
\usepackage[portuguese]{babel}

% Enables code listing
\usepackage{listings}
%--------------------------------------
\usepackage[T1]{fontenc}
\usepackage[utf8]{inputenc}

\usepackage[figuresright]{rotating}
\usepackage{amsthm}
\usepackage{graphics}
\usepackage{amssymb}
\usepackage{graphicx}
\usepackage{fancybox}
\usepackage{amsmath}
\usepackage{picinpar}
\usepackage{colortbl}
\usepackage{wasysym}
\usepackage{txfonts}
\usepackage{pb-diagram}
\usepackage{relsize}
\usepackage{tikz}
	\usetikzlibrary{calc}
	\usetikzlibrary{datavisualization}
	\usetikzlibrary{positioning}
	\usetikzlibrary{mindmap}
	\usetikzlibrary{snakes}
	\usetikzlibrary{shapes}
	\usetikzlibrary{decorations.pathreplacing}
	\usetikzlibrary{spy}
	\usetikzlibrary{backgrounds}
	\usetikzlibrary{patterns}
\usepackage{pgfplots}
\usepackage{pgfplotstable}
	\pgfplotsset{compat=newest}
	\usepgfplotslibrary{units}
\usepackage{subfigure}
\usepackage{algorithm}
\usepackage{algorithmic}
\usepackage{verbatim}
\usepackage{wrapfig}
\usepackage{array}
\usepackage{calc}
\usepackage[T1]{fontenc}
\usepackage{times}
\usepackage{indentfirst}        % indenta primeiro par�grafo
\usepackage{fancyhdr}
\usepackage{pifont}
\usepackage{textcomp}      % \texttrademark
\usepackage{url}  
\usepackage{multirow}  
\usepackage[numbers]{natbib}
\usepackage{notoccite}
\usepackage{setspace}
\usepackage{array}
\usepackage{helvet}
\renewcommand{\familydefault}{\sfdefault}
\headheight 16pt
\setlength{\topmargin}{-15pt} % extra vert. space + at the top of header: 23pt
\setlength{\oddsidemargin}{0pt} % extra spc added at the left of odd page: 0pt
\setlength{\evensidemargin}{-12pt} % ext. spc added at the left of even pg: 59pt
\setlength{\textheight}{638pt} % height of the body: 592pt
\setlength{\textwidth}{483pt} % width of the body: 470pt
\pagestyle{fancyplain}
\renewcommand{\chaptermark}[1]{\markboth{#1}{}}
\renewcommand{\sectionmark}[1]{\markright{\thesection\ #1}}
\lhead[\fancyplain{}{\bfseries\thepage}]{\fancyplain{}{\bfseries\rightmark}}
\rhead[\fancyplain{}{\bfseries\leftmark}]{\fancyplain{}{\bfseries\thepage}}
\cfoot[\fancyplain{\bfseries\thepage}{}]{\fancyplain{\bfseries\thepage}{}}
\newenvironment{myenv}[1]
  {\begin{spacing}{#1}}
  {\end{spacing}}

%% Inicia o texto
\begin{document}
	%% Abrevia figuras e tabelas
	\thispagestyle{empty}
	\begin{center}
		\par \null
		\begin{figure}[H]
			\centering \includegraphics[angle=-90]{unesp.pdf}
		\end{figure} 
		\vspace{3cm}
		\fontsize{14}{\baselineskip} \selectfont
		{André Furlan} \\  
		\vspace{4.5cm}
		\onehalfspacing
		\fontsize{14}{\baselineskip} \selectfont
		Monografia de estudos especiais \\
		\vspace{7cm}
		\fontsize{14}{\baselineskip} \selectfont  
		{São José do Rio Preto}\\ \vspace{1.0pt} 
		{2019} 
	\end{center}

	%% *****ROSTO*****
	\newpage
	\thispagestyle{empty}
	\setcounter{page}{1}
	\begin{center}
		\vspace{4cm}
		\fontsize{14}{\baselineskip} \selectfont
		\vspace{30.0pt}
		{André Furlan} \\ \vspace{30.0pt}
		{Caracterização de \textit{voice spoofing} para fins de verificação de locutores com base na transformada wavelet e na análise paraconsistente de características} \\ \onehalfspacing \fontsize{14}{\baselineskip}
		\par \null
		\begin{flushright}
		\parbox{3.50in}{
			\fontsize{12}{\baselineskip} \selectfont \onehalfspacing
			Monografia apresentada para cumprimento da disciplina de estudos especiais do curso de Mestrado em Ciência da Computação, junto ao Programa de Pós-Graduação em Ciência da Computação, do Instituto de Biociências, Letras e Ciências Exatas da Universidade Estadual Paulista "Júlio de Mesquita Filho", Campus de São José do Rio Preto. \\ \vspace{1.0pt}
			{Orientador: Prof. Dr. Rodrigo Capobianco Guido } \\ \vspace{1.0pt}
		}
		\end{flushright}
		\fontsize{14}{\baselineskip} \selectfont
		\vspace{8.0cm}
		São José do Rio Preto, SP \\ \vspace{1.0pt}  
		2019
	\end{center}
	\newpage
	\thispagestyle{empty}

	%% *****APROVACAO*****
	\begin{center}
		\vspace{4cm}
		\fontsize{14}{\baselineskip} \selectfont
		\vspace{30.0pt}
		{André Furlan} \\ \vspace{30.0pt}
		{Caracterização de \textit{voice spoofing} para fins de verificação de locutores com base na transformada wavelet e na análise paraconsistente de características} \\ \onehalfspacing \fontsize{14}{\baselineskip} \selectfont
		\par \null
		\begin{flushright}
		\parbox{3.50in}{
			\fontsize{12}{\baselineskip} \selectfont \onehalfspacing
			Monografia apresentada para cumprimento da disciplina de estudos especiais do curso de Mestrado em Ciência da Computação, junto ao Programa de Pós-Graduação em Ciência da Computação, do Instituto de Biociências, Letras e Ciências Exatas da Universidade Estadual Paulista "Júlio de Mesquita Filho", Campus de São José do Rio Preto. \\ \vspace{1.0pt}
			{Orientador: Prof. Dr. Rodrigo Capobianco Guido } \\ \vspace{1.0pt}
		}
		\end{flushright}
		\fontsize{14}{\baselineskip} \selectfont
		Comissão Examinadora \\  \vspace{1.0pt}
	\end{center}

	\fontsize{14}{\baselineskip} \selectfont
	Prof. Dr. Rodrigo Capobianco Guido \\ 
	UNESP - Campus de São José do Rio Preto \\
	Co-Orientador \\\\
	
	Prof Dr xxxxxxxx \\ 
	UNESP - Campus de São José do Rio Preto \\\\
	
	Prof Dr xxxxx \\
	xxxxxxxx \\
	\vspace{3.0cm}

	\begin{center}
		São José do Rio Preto, SP  \\ \vspace{1.0pt}
		2019
	\end{center}

	%%
	%% *****Resumo*****
	%%

	%% Resumo/Abstract
	%% *****Resumo*****
	\setlength{\parindent}{0pt}
	\newpage \thispagestyle{empty}
	\vspace{1.5cm}
	\fontsize{12}{\baselineskip} \selectfont

	\begin{center}
		{\huge{\textbf{RESUMO}}}
	\end{center}

	\begin{myenv}{1.5}
		\fontsize{12}{\baselineskip} \selectfont \onehalfspacing
		\par \null
		\par \null
		Este documento constitui uma monografia da disciplina de estudos especiais contendo levantamento bibliográfico dos estudos já realizados.
	\end{myenv}

	%% Lista de figuras (gerada automaticamente)
	\cleardoublepage
	\pagenumbering{gobble}
	\listoffigures
	
	% Lista de tabelas (gerada automaticamente)
	\cleardoublepage
	\pagenumbering{gobble}
	\listoftables
	\frontmatter
	
	%% Lista de conte\'{u}do (sumário)
	\def\contentsname{Sumário} 
	\pagenumbering{gobble}
	\tableofcontents
	\cleardoublepage
	
	% \nobibintoc para bibliografia nao aparecer no indice
	%% Glossário (gerado automaticamente - veja entradas em cap1.tex)
	%\cleardoublepage
	%\renewcommand{\nomname}{Glossário}
	%\markboth{GLOSSáRIO}{GLOSSáRIO}
	%\addcontentsline{toc}{chapter}{\nomname}
	%\printnomenclature
	\mainmatter
	\setlength{\parindent}{1.25cm}

	\chapter{Introdução}
		\begin{myenv}{1.5}
			\setcounter{page}{12}
			Os sistemas de reconhecimento biométrico podem se basear em diversos aspectos do usuário neste caso será discutida, especificamente, a voz e sua respectiva tentativa de falseamento usando o processo de \textit{voice spoofing} e como se pretende reconhecer tal ataque.
			Um sistema de reconhecimento \textbf{idealmente} não deve se deixar enganar por, como exemplo, uma voz gravada.\\
			Neste documento será apresentada uma revisão de conceitos sobre os seguintes tópicos:
			\begin{itemize}
				\item Engenharia paraconsistente.
				\item Filtros digitais usando wavelets.
				\item Caracterização dos processos de produção da voz humana.
				\item Amostragem, quantização, entre outros.
			\end{itemize}
			Em seguida apresentar-se-á uma revisão bibliográfica finalizando com o cronograma previsto.
		\end{myenv}

	\chapter{Revisão de conceitos}
		\begin{myenv}{1.5}
			\section{Lógica paraconsistente}
			\par Quando se trata de medir fenômenos até dentro de um domínio ideal os mesmos podem vir com uma certa dose de incerteza. No contexto da análise de sinais, dadas a similaridades e ruídos presentes esta ambiguidade tende a ficar ainda mais evidente.
			\par Já que a lógica clássica tem dificuldade de lidar com tais cenários eis que a paraconsistente entra.
			\par Nesta abordagem é possível se definir \textit{"níveis de verdade"} dado um valor a ser medido. Por exemplo, quão próximo de um adulto ou uma criança está alguém de 17 anos? Ou ainda, qual a proximidade de uma vogal dita por um locutor do som de \textbf{"e"} ou \textbf{"i"}? 
			
			\section{Filtros digitais usando wavelets}
			\section{Caracterização dos processos de produção da voz humana}
			\section{Amostragem, quantização}
		
		\end{myenv}

	\chapter{Revisão Bibliográfica}
		\begin{myenv}{1.5}
			\par 
		\end{myenv}
	
	\label{c_rl}
	\begin{myenv}{1.5}
		\textit{Neste Capítulo....}
		\section{XXXXX}
		\par Os sinais.... de acordo com a Figura \ref{sinaldigital}, observa-se que....
		
		\begin{figure}
			\centering 
			\begin{tikzpicture} 
					\begin{axis}[height=3cm,width=9cm,xlabel=tempo,ylabel=amplitude, xmin=-1,xmax=19, xtick={0,9,18}, ymin=-0.1,ymax=1.1,ytick={0,1}]
					\addplot[gray,dotted,mark=none,domain=-1:19,samples=2]{0};
					\addplot[brown] coordinates{
					(0,0.08208)(1,0.10316)(2,0.14718)(3,0.21919)(4,0.26128)(5,0.31235)(6,0.36336)(7,0.42241)(8,0.48649)(9,0.53854)(10,0.58855)(11,0.63360)(12,0.69369)(13,0.73272)(14,0.77173)(15,0.81879)(16,0.85689)(17,0.88891)(18,0.91892)
					};
					\addplot[green] coordinates{
					(0,0.02202)(1,0.07007)(2,0.10210)(3,0.16016)(4,0.17117)(5,0.17818)(6,0.18418)(7,0.23323)(8,0.28128)(9,0.34735)(10,0.36036)(11,0.36737)(12,0.44244)(13,0.52953)(14,0.54555)(15,0.60160)(16,0.71471)(17,0.73473)(18,0.89890)
					};
				\end{axis}
			\end{tikzpicture}
			\caption{representação de dois sinais digitais.}
			\label{sinaldigital}
		\end{figure}
		\par Assim sendo.... de acordo com o artigo \cite{mllp}, observa-se que....
	\end{myenv}


	


	\bibliography{bibliography.bib}
	\bibliographystyle{alpha}

\end{document}

