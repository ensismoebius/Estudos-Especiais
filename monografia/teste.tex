\documentclass[a4paper,12pt,openright,oneside]{book}

% Enables Portuguese Brasil
\usepackage[portuguese]{babel}

% Enables code listing
\usepackage{listings}
%--------------------------------------

\usepackage{xcolor}

%--------------------------------------

\usepackage[T1]{fontenc}
\usepackage[utf8]{inputenc}

\usepackage[figuresright]{rotating}
\usepackage{amsthm}
\usepackage{graphics}
\usepackage{amssymb}
\usepackage{graphicx}
\usepackage{fancybox}
\usepackage{amsmath}
\usepackage{picinpar}
\usepackage{colortbl}
\usepackage{wasysym}
\usepackage{txfonts}
\usepackage{pb-diagram}
\usepackage{relsize}
\usepackage{tikz}
\usetikzlibrary{calc}
\usetikzlibrary{datavisualization}
\usetikzlibrary{positioning}
\usetikzlibrary{mindmap}
\usetikzlibrary{snakes}
\usetikzlibrary{shapes}
\usetikzlibrary{decorations.pathreplacing}
\usetikzlibrary{spy}
\usetikzlibrary{backgrounds}
\usetikzlibrary{patterns}
\usepackage{pgfplots}
\usepackage{pgfplotstable}
\pgfplotsset{compat=newest}
\usepgfplotslibrary{units}
\usepackage{subfigure}
\usepackage{algorithm}
\usepackage{algorithmic}
\usepackage{verbatim}
\usepackage{wrapfig}
\usepackage{array}
\usepackage{calc}
\usepackage[T1]{fontenc}
\usepackage{times}
\usepackage{indentfirst}        % indenta primeiro par�grafo
\usepackage{fancyhdr}
\usepackage{pifont}
\usepackage{textcomp}      % \texttrademark
\usepackage{url}  
\usepackage{multirow}  
\usepackage[numbers]{natbib}
\usepackage{notoccite}
\usepackage{setspace}
\usepackage{array}
\usepackage{helvet}

\renewcommand{\familydefault}{\sfdefault}
\headheight 16pt
\setlength{\topmargin}{-15pt} % extra vert. space + at the top of header: 23pt
\setlength{\oddsidemargin}{0pt} % extra spc added at the left of odd page: 0pt
\setlength{\evensidemargin}{-12pt} % ext. spc added at the left of even pg: 59pt
\setlength{\textheight}{638pt} % height of the body: 592pt
\setlength{\textwidth}{483pt} % width of the body: 470pt
\pagestyle{fancyplain}
\renewcommand{\chaptermark}[1]{\markboth{#1}{}}
\renewcommand{\sectionmark}[1]{\markright{\thesection\ #1}}
\lhead[\fancyplain{}{\bfseries\thepage}]{\fancyplain{}{\bfseries\rightmark}}
\rhead[\fancyplain{}{\bfseries\leftmark}]{\fancyplain{}{\bfseries\thepage}}
\cfoot[\fancyplain{\bfseries\thepage}{}]{\fancyplain{\bfseries\thepage}{}}
\newenvironment{myenv}[1]
{\begin{spacing}{#1}}
	{\end{spacing}}




%% Inicia o texto

\begin{document}

	%% Abrevia figuras e tabelas

	\thispagestyle{empty}

	\begin{center}

		\par \null

		\begin{figure}[H]

			\centering \includegraphics[angle=-90]{unesp.pdf}

		\end{figure} 

		\vspace{3cm}

		\fontsize{14}{\baselineskip} \selectfont

		{André Furlan} \\  

		\vspace{4.5cm}

		\onehalfspacing

		\fontsize{14}{\baselineskip} \selectfont

		Monografia de estudos especiais \\

		\vspace{7cm}

		\fontsize{14}{\baselineskip} \selectfont  

		{São José do Rio Preto}\\ \vspace{1.0pt} 

		{2019} 

	\end{center}




	%% *****ROSTO*****

	\newpage

	\thispagestyle{empty}

	\setcounter{page}{1}

	\begin{center}

		\vspace{4cm}

		\fontsize{14}{\baselineskip} \selectfont

		\vspace{30.0pt}

		{André Furlan} \\ \vspace{30.0pt}

		{Caracterização de \textit{voice spoofing} para fins de verificação de locutores com base na transformada wavelet e na análise paraconsistente de características} \\ \onehalfspacing \fontsize{14}{\baselineskip}

		\par \null

		\begin{flushright}

		\parbox{3.50in}{

			\fontsize{12}{\baselineskip} \selectfont \onehalfspacing

			Monografia apresentada para cumprimento da disciplina de estudos especiais do curso de Mestrado em Ciência da Computação, junto ao Programa de Pós-Graduação em Ciência da Computação, do Instituto de Biociências, Letras e Ciências Exatas da Universidade Estadual Paulista ``Júlio de Mesquita Filho’’, Campus de São José do Rio Preto. \\ \vspace{1.0pt}

			{Orientador: Prof. Dr. Rodrigo Capobianco Guido } \\ \vspace{1.0pt}

		}

		\end{flushright}

		\fontsize{14}{\baselineskip} \selectfont

		\vspace{8.0cm}

		São José do Rio Preto, SP \\ \vspace{1.0pt}  

		2019

	\end{center}

	\newpage

	\thispagestyle{empty}




	%% *****APROVACAO*****

	\begin{center}

		\vspace{4cm}

		\fontsize{14}{\baselineskip} \selectfont

		\vspace{30.0pt}

		{André Furlan} \\ \vspace{30.0pt}

		{Caracterização de \textit{voice spoofing} para fins de verificação de locutores com base na transformada \textit{wavelet} e na análise paraconsistente de características} \\ \onehalfspacing \fontsize{14}{\baselineskip} \selectfont

		\par \null

		\begin{flushright}

		\parbox{3.50in}{

			\fontsize{12}{\baselineskip} \selectfont \onehalfspacing

			Monografia apresentada para cumprimento da disciplina de estudos especiais do curso de Mestrado em Ciência da Computação, junto ao Programa de Pós-Graduação em Ciência da Computação, do Instituto de Biociências, Letras e Ciências Exatas da Universidade Estadual Paulista ``Júlio de Mesquita Filho’’, Campus de São José do Rio Preto. \\ \vspace{1.0pt}

			{Orientador: Prof. Dr. Rodrigo Capobianco Guido } \\ \vspace{1.0pt}

		}

		\end{flushright}

		\fontsize{14}{\baselineskip} \selectfont

		Comissão Examinadora \\  \vspace{1.0pt}

	\end{center}




	\fontsize{14}{\baselineskip} \selectfont

	Professor Dr. Rodrigo Capobianco Guido \\ 

	UNESP - Campus de São José do Rio Preto \\

	Co-Orientador \\\\

	

	Professora Dra. Renata Spolon Lobato \\ 

	UNESP - Campus de São José do Rio Preto \\\\

	

	Professor Dr. Aleardo Manacero Júnior \\

	UNESP - Campus de São José do Rio Preto \\

	\vspace{1.0cm}




	\begin{center}

		São José do Rio Preto, SP  \\ \vspace{1.0pt}

		2019

	\end{center}
%%
%% *****Resumo*****
%%
%% Resumo/Abstract
%% *****Resumo*****
\setlength{\parindent}{0pt}
\newpage \thispagestyle{empty}
\vspace{1.5cm}
\fontsize{12}{\baselineskip} \selectfont

\begin{center}
{\huge{\textbf{RESUMO}}}
\end{center}

\begin{myenv}{1.5}
\fontsize{12}{\baselineskip} \selectfont \onehalfspacing
\par \null
\par \null
\par Este documento constitui a monografia produzida como resultado dos estudos especiais realizados pelo autor, visando promover um levantamento bibliográfico inicial do tema de sua dissertação de mestrado. Foram inclusas as descrições essenciais de 15 trabalhos científicos na área de \textit{voice spoofing detection}, acrescidos ainda, das direções que caracterizam o trabalho em questão, visando a confecção da monografia de qualificação. No Capítulo 1 é realizada uma breve introdução. Em seguida, no Capítulo 2, são iniciadas as revisões de conceitos, englobando a engenharia paraconsistente de características, os filtros digitais \textit{wavelet}, os conceitos de amostragem e de quantização, além da descrição do formato \textit{wave} de arquivos e da caracterização dos processos de produção da voz humana. O Capítulo é encerrado com uma revisão dos trabalhos correlatos que foram selecionados pelo autor. Finalmente, no Capítulo 3, é disponibilizado o cronograma do trabalho de mestrado do autor, com as etapas já realizadas e uma previsão da realização das próximas.
\end{myenv}

%% Lista de figuras (gerada automaticamente)
\cleardoublepage
\pagenumbering{gobble}
\listoffigures
% Lista de tabelas (gerada automaticamente)
\cleardoublepage
\pagenumbering{gobble}
\listoftables
\frontmatter
%% Lista de conte\'{u}do (sumário)
\def\contentsname{Sumário} 
\pagenumbering{gobble}
\tableofcontents
\cleardoublepage
% \nobibintoc para bibliografia não aparecer no índice
%% Glossário (gerado automaticamente - veja entradas em cap1.tex)
%\cleardoublepage
%\renewcommand{\nomname}{Glossário}
%\markboth{GLOSSáRIO}{GLOSSáRIO}
%\addcontentsline{toc}{chapter}{\nomname}
%\printnomenclature
\mainmatter
\setlength{\parindent}{1.25cm}

\chapter{Introdução}
\begin{myenv}{1.5}
\setcounter{page}{12}
\par \textbf{Idealmente}, um sistema de autenticação biométrica de locutores não deve se deixar enganar por uma voz gravada do locutor-alvo, por exemplo. Assim, os \textit{voice spoofing attacks} do tipo \textit{playback speech} constituem o tema de estudo deste trabalho, por meio do qual uma revisão dos seguintes conceitos é apresentada: engenharia paraconsistente de características, filtros digitais do tipo \textit{wavelet}, processo de produção da voz humana, amostragem, quantização, entre outros, além de trabalhos correlatos. Ao final, o cronograma das atividades foi também incluso.
\end{myenv}

\chapter{Revisão de Bibliográfica}
\begin{myenv}{1.5}
\section{Conceitos utilizados}
\subsection{Engenharia paraconsistente de características}
\par Em meio aos processos de reconhecimento de padrões, frequentemente surge a questão: ``Os vetores de características associados aos sinais sob análise proporcionam uma boa separação entre as classes?’’. Para respondê-la, a engenharia paraconsistente de características, que é uma estratégia recém publicada, pode ser usada. 
\\
\par O processo inicia-se após a aquisição dos vetores de características para cada classe ($C_n$), para $n=1, 2, 3, ...$. É necessária a normalização dos vetores de forma que todos os seus componentes estejam na faixa de $0$ até $1$.
\\
\par Em seguida, o cálculo de duas grandezas deve ser realizado:
\begin{itemize}
\item a menor similaridade intraclasse ($\alpha$)
\item a razão de sobreposição interclasse ($\beta$). 
\end{itemize}
\par Particularmente, $\alpha$ indica o nível de similaridade que os vetores têm entre si, dentro de uma mesma classe. Diferentemente, $\beta$ reflete a taxa de sobreposição entre diferentes classes. Idealmente, $\alpha$ deve ser maximizada e $\beta$ minimizada para uma acurácia ótima do classificador que será usado para reconhecer os padrões em questão.
\\
\par Em seguida, os maiores e os menores valores de cada uma das posições dos vetores de características de cada classe são obtidos, gerando assim dois novos vetores: um para os maiores valores e outro para os menores. O \textbf{vetor de similaridade} da classe, ($svC_n$), é então obtido fazendo-se a diferença, item a item, dos maiores em relação aos menores elementos. Finalmente, para cada classe, é calculada a média dos valores de cada vetor de similaridade, sendo que $\alpha$ é o menor valor dentre essas médias. Todo o processo está ilustrado na figura \ref{fig:calculoalpha}.
\begin{figure}[H]
\centering
\includegraphics[width=0.6\linewidth]{images/calculoAlpha.pdf}
\caption{cálculo de $\alpha$.}
\label{fig:calculoalpha}
\end{figure}
\par A obtenção de $\beta$, assim como ilustrado na figura \ref{fig:betacalculation}, também se dá selecionando-se os maiores e os menores valores de cada uma das posições de todos os vetores de características de cada classe, gerando assim um vetor para os valores maiores e outro para os menores. Na sequência, ocorre o cálculo de $R$, que corresponde a quantidade de vezes que um valor do vetor de características de uma classe se encontra no intervalo de valores maiores e menores de outra classe.
\\
\par É ainda necessário o cálculo de $F$, que é o número máximo de sobreposições possíveis entre classes, sendo dado por
\begin{equation}
F=N \cdot (N-1) \cdot X \cdot T \qquad, 
\end{equation}
onde:
\begin{itemize}
\item $N$ é a quantidades de classes;
\item $X$ é quantidade de vetores de características por classe;
\item $T$ é o tamanho do vetor de características.
\end{itemize}
\par Finalmente, $\beta$ é calculado da seguinte forma:
\begin{equation}
\beta=\dfrac{R}{F}
\end{equation}
\par Neste ponto, é importante notar que $\alpha=1$ sugere fortemente que os vetores de características de cada classe são similares e representam suas respectivas classes precisamente. Complementarmente, $\beta=0$ sugere os vetores de características de classes diferentes não se sobrepõe \cite{8588433}.
\begin{figure}[H]
\centering
\includegraphics[width=0.55\linewidth]{images/betaCalculation.pdf}
\caption{cálculo de $\beta$}
\label{fig:betacalculation}
\end{figure}
De posse de $\alpha$ e $\beta$, estabelece-se o plano paraconsistente, no qual:
\begin{itemize}
\item Verdade $\rightarrow$ fé total ($\alpha = 1$) e nenhum descrédito ($\beta = 0$);
\item Ambiguidade $\rightarrow$ fé total ($\alpha = 1$) e descrédito total ($\beta = 1$);
\item Falsidade $\rightarrow$ fé nula ($\alpha = 0$) e descrédito total ($\beta = 1$);
\item Indefinição $\rightarrow$ fé nula ($\alpha = 0$) e descrédito total ($\beta = 0$).
\end{itemize}
\par No entanto, raramente $\alpha$ e $\beta$ terão valores extremos. Na maioria das vezes, $0 < \alpha < 1$ e $0 < \beta < 1$. Por isso, se torna necessário o cálculo do \textbf{grau de certeza}($G_1$) e do \textbf{grau de contradição}($G_2$), como segue:
\begin{equation}
G_1=\alpha-\beta\qquad,
\end{equation}
\begin{equation}
G_2=\alpha+\beta-1 \qquad,
\end{equation}
onde $-1 \leqslant G_1 \leqslant 1$ e  $-1 \leqslant G_2 \leqslant 1$.
\\
\par Os valores de $G_1$ e $G_2$, em conjunto, definem os graus entre verdade e falsidade, ou seja, $G_1=-1$ e $G_1=1$, respectivamente, e também os graus entre indefinição e ambiguidade, ou seja, $G_2=-1$ e $G_2=1$, respectivamente.
\\
\par O plano paraconsistente para fins de visualização e maior rapidez na avaliação dos resultados, como ilustrado na figura \ref{fig:paraconsistentplane}, possui quatro vértices definidos:
\begin{itemize}
\item (-1,0) $\rightarrow$ Falsidade;
\item (1,0) $\rightarrow$ Verdade;
\item (0,-1) $\rightarrow$ Indefinição;
\item (0,1) $\rightarrow$ Ambiguidade.
\end{itemize}
\par É importante perceber que na figura \ref{fig:paraconsistentplane} existe um pequeno círculo indicando onde se encontram as classes nos graus explicados da listagem anterior. Para se ter ideia em que área exatamente se encontram as classes avaliadas se deve calcular as distâncias$(D)$ do ponto $P=(G_1,G_2)$ dos limites supracitados. Tal cálculo pode ser feito da seguinte forma:
\begin{equation}
D_{-1,0}=\sqrt{(G_1+1)^2+(G_2)^2}\qquad,
\end{equation}
\begin{equation}
D_{1,0}=\sqrt{(G_1-1)^2+(G_2)^2}\qquad,
\end{equation}
\begin{equation}
D_{0,-1}=\sqrt{(G_1)^2+(G_2+1)^2}\qquad,		
\end{equation}
\begin{equation}
D_{0,1}=\sqrt{(G_1)^2+(G_2-1)^2}\qquad,
\end{equation}		
\begin{figure}[H]
\centering					\includegraphics[width=0.69\linewidth]{images/paraconsistentPlane.pdf}
\caption{o plano paraconsistente.}
\label{fig:paraconsistentplane}
\end{figure}
\subsection{Filtros digitais \textit{wavelet}}
\par Filtros digitais \textit{wavelet} podem ser usados para suprir as deficiências de janelamento apresentadas pela Transformada de Fourier e pela Transformada de Fourier de Tempo Reduzido. \textit{Wavelets} contam com variadas funções-filtro e tem tamanho de janela variável o que permite uma análise multirresolução \cite{Rod5254905}. As \textit{wavelets} proporcionam a análise do sinal de forma detalhada tanto no espectro de baixa frequência quanto no de alta frequência.
\\
\par É importante observar que, quando se trata de Transformada \textit{Wavelet}, seis elementos estão presentes: dois filtros de análise, dois filtros de síntese e as funções ortogonais \textit{scaling} e \textit{wavelet}. Tendo em vista que somente a transformada direta, e não a inversa, será usada na construção dos vetores de características, os filtros de síntese, a função \textit{scaling} e a função \textit{wavelet} não serão elementos abordados aqui, pois, eles só interessariam caso houvesse a necessidade de transformações inversas. Assim, a abordagem usada será baseada nos filtros de análise que proporcionarão a decomposição do sinal com o uso de respostas passa-baixas e passa-altas, estritamente no domínio discreto.
\\
\par No contexto dos filtros digitais baseados em \textit{wavelets}, o tamanho da janela recebe o nome de \textbf{suporte}. Janelas definem o tamanho do filtro que será aplicado ao sinal e, quando ele é pequeno, se diz que a janela tem \textbf{um suporte compacto} \cite{robi2003}.
\\
\par Em termos de respostas em frequência, as wavelets de Daubechies se destacam por serem \textit{maximamente planas} (\textit{maximally-flat}) nos platôs de resposta em frequência, isto é, nas bandas de passagem e rejeição, como indicado na figura \ref{fig:daubechies} e diferente da figura \ref{fig:nomaximallyflat}.
\begin{figure}[H]
\centering
\includegraphics[width=0.3\linewidth]{images/daubechies}
\caption{platôs maximamente planos em um filtro digital}
\label{fig:daubechies}
\end{figure}
\begin{figure}[H]
\centering
\includegraphics[width=0.3\linewidth]{images/noMaximallyFlat}
\caption{platôs não maximamente planos de um filtro digital}
\label{fig:nomaximallyflat}
\end{figure}
\par Além da resposta em frequência, a aplicação de um filtro digital \textit{wavelet} também deve considerar a \textbf{resposta em fase}, implicando o deslocamento \textbf{linear}, \textbf{quase linear} ou \textbf{não linear} dos componentes de frequências restantes após a filtragem. Particularmente:
\begin{itemize}
\item na resposta em fase \textbf{linear}, há o mesmo deslocamento de fase para todos os componentes do sinal;
\item na resposta em fase \textbf{quase linear}, existe uma pequena diferença no deslocamento dos diferentes componentes espectrais do sinal;
\item na resposta \textbf{não-linear}, acontece um deslocamento significativamente heterogêneo para as diferentes frequências restantes no sinal.
\end{itemize}
\par Idealmente, é desejável que todo filtro apresente boa resposta em frequência e resposta em fase linear. A tabela \ref{tab:waveletsProperties} contém detalhes nesse sentido.
\begin{table}[H]
\centering
\begin{tabular}{|c|c|c|}
\hline 
\textbf{Wavelet} & \textbf{Resposta em frequência} & \textbf{Resposta em fase} \\ 
\hline 
Haar & Pobre &  Linear \\ 
\hline 
Daubechies & Quanto maior o suporte, melhor. \textit{Maximally-flat}  &  Não linear \\ 
\hline 
Symmlets & Quanto maior o suporte, melhor. Não \textit{Maximally-flat} & Quase linear \\ 
\hline 
Coiflets & Quanto maior o suporte, melhor. Não \textit{Maximally-flat} & Quase linear \\ 
\hline 
\end{tabular} 
\caption{algumas \textit{wavelets} mais populares e suas propriedades.}	\label{tab:waveletsProperties}
\end{table}
\subsubsection{O algoritmo de Mallat}
\par O algoritmo de Mallat torna aplicação das \textit{wavelets} uma simples multiplicação de matrizes. O sinal que deve ser transformado se torna uma matriz linear vertical; já os filtros passa-baixa e passa-alta tornam-se, nessa ordem, linhas de uma matriz quadrada que será completada segundo regras que serão mostradas mais adiante.
\\
\par É importante que essa matriz quadrada tenha de aresta a mesma quantidade de itens que o nosso sinal, ou seja, se, por exemplo, o sinal tem quatro elementos então a matriz de filtros deve ser 4x4. Algo interessante a se notar é que, para que seja possível a transformadação, basta ter disponível o filtro passa-baixas, pois o filtro passa-altas pode ser construído a partir da ortogonalidade do primeiro.
\\
\par A título de exemplo considere o filtro passa baixa baseado na \textit{wavelet} Haar: $h[\cdot] = [\frac{1}{\sqrt{2}}, \frac{1}{\sqrt{2}}]$ e o seu respectivo par ortogonal: $g[\cdot] = [\frac{1}{\sqrt{2}}, \frac{-1}{\sqrt{2}}].$ Considere também o seguinte sinal: $s[\cdot] = [1,2,3,4].$
\\
\par Se o tamanho do sinal a ser tratado é quatro, ou seja, o sinal tem quatro amostras, e se pretende-se aplicar o filtro Haar, a seguinte matriz é construída:
\begin{equation}
\begin{pmatrix}
\frac{1}{\sqrt{2}}, \frac{1}{\sqrt{2}}, 0, 0\\
\frac{1}{\sqrt{2}}, \frac{-1}{\sqrt{2}}, 0, 0\\
0, 0, \frac{1}{\sqrt{2}}, \frac{1}{\sqrt{2}}\\
0, 0, \frac{1}{\sqrt{2}}, \frac{1}{\sqrt{2}}\\
\end{pmatrix} 
\end{equation}
\par No entanto, filtros Haar tem apenas dois valores e, necessariamente, a linha da matriz deve ter quatro itens. Para resolver este problema basta completar cada uma das linhas com zeros. A matriz é montada de forma que ela seja ortogonal. Definida a matriz de filtros, segue-se com os cálculos da transformada:
\begin{equation}
\begin{pmatrix}
\frac{1}{\sqrt{2}}, \frac{1}{\sqrt{2}}, 0, 0\\
\frac{1}{\sqrt{2}}, \frac{-1}{\sqrt{2}}, 0, 0\\
0, 0, \frac{1}{\sqrt{2}}, \frac{1}{\sqrt{2}}\\
0, 0, \frac{1}{\sqrt{2}}, \frac{1}{\sqrt{2}}\\
\end{pmatrix} 
\cdot
\begin{pmatrix}
1\\
2\\
3\\
4\\
\end{pmatrix} 
=
\begin{pmatrix}
\frac{3}{\sqrt{2}}\\
\frac{-1}{\sqrt{2}}\\
\frac{7}{\sqrt{2}}\\
\frac{-1}{\sqrt{2}}\\
\end{pmatrix}
\end{equation}
\par Realizada a multiplicação, é necessário extrair do resultado o sinal filtrado. Isso é feito escolhendo, dentro do resultado, valores alternadamente de forma que o vetor resultante seja:
\begin{equation}
resultado = \Big[
\frac{3}{\sqrt{2}},
\frac{7}{\sqrt{2}},
\frac{-1}{\sqrt{2}},
\frac{-1}{\sqrt{2}}
\Big]\qquad.
\end{equation}

\subsection{Amostragem, quantização e o formato do arquivo Wave}
\par Neste trabalho, serão usados arquivos no formato \textit{wave} usando \textit{pulse-code modulation} (PCM). Neles, os dados são armazenados sem perdas. O arquivo, segundo \cite{WAVE2019}, se estrutura como o ilustrado na figura \ref{fig:wavePcmStructure}. A taxa de amostragem de 44100hz permite, segundo o teorema de Nyquist, que seja realizada a quantização de frequências de até 22050hz a uma resolução de 16bits.
\begin{figure}[H]
\centering
\includegraphics[width=0.45\linewidth, angle=-90]{images/wavePcmStructure.pdf}
\caption{estrutura do arquivo \textit{wave}.}
\label{fig:wavePcmStructure}
\end{figure}
\par A estrutura de interesse se localiza na última parte do arquivo, mais especificamente no bloco ``data''. Nele, os dados são organizados como um grande vetor de números, cada um deles indicando a intensidade do sinal naquele ponto.
\subsection{Caracterização dos processos de produção da voz humana}
\par A fala possui três grandes áreas de estudo: fisiológica, ou fonética articulatória, acústica, ou fonética acústica, e  perceptual,  comumente  chamada percepção  da  fala \cite{kremer2014eficiencia}. Neste trabalho, o foco será apenas na acústica, já que não serão analisados aspectos da fisiologia relacionada a voz e sim o sinal sonoro propriamente dito.
\subsubsection{Emissão sonora vozeada versus não-vozeada}
\par Quando da análise da voz, se pode levar em consideração as partes vozeadas e/ou não-vozeadas do sinal. As partes vozeadas são aquelas produzidas com ajuda do movimento quase-periódico das pregas vocais; as partes não-vozeadas não tem participação dessa estrutura.
\subsubsection{Frequência fundamental da voz}
\par Também conhecida como $F_0$, é o componente periódico resultante da vibração das pregas vocais. Em termos de percepção, pode-se interpretar $F_0$ como o tom da voz, ou seja, o \textit{pitch} \cite{kremer2014eficiencia}. Vozes agudas tem um \textit{pitch} alto, vozes mais graves tem um \textit{pitch} baixo, sendo que a alteração do \textit{pitch} durante a fala é definido como entonação \cite{freitas2013avaliaccao}. A medição de $F_0$ está sujeita a contaminações surgidas das variações naturais de \textit{pitch} típicas da voz humana \cite{freitas2013avaliaccao}. 
\subsubsection{Formantes}
\par \textit{O primeiro formante ($F_1$), relaciona-se à  amplificação  sonora  na  cavidade  oral  posterior  e  à  posição  da  língua  no  plano  vertical;  o segundo  formante  ($F_2$)  à  cavidade  oral  anterior  e  à  posição  da  língua  no  plano  horizontal; o terceiro  formante  ($F_3$)  relaciona-se  às  cavidades  à  frente  e  atrás  do  ápice  da  língua;  o  quarto formante  ($F_4$),  ao  formato  da  laringe  e  da  faringe  na  mesma  altura} \cite{valencca2014analise}.


				%\subsubsection{Duvidas}

				%	\par Como o tamanho da janela funciona em wavelet? Pensei que o tamanho da janela era dado pela função de escalamento da wavelet.

				%	\par Como eu defino o tamanho da janela no algoritmo de Malat?	

				%	\par O número de sobre posições no cálculo de R é de valor ou de intervalo?

\section{Trabalhos correlatos}
\par No artigo \cite{Ren2019}, foi apresentado um esquema de diferenciação entre a fala comum e aquela vinda de um dispositivo reprodutor. O foco da análise se dá na distorção causada pelo alto-falante segundo a energia e outras várias características do espectro do sinal. Uma base com 771 sinais de fala foi criada para cada um dos quatro dispositivos de gravação usados, totalizando 3084 trechos de áudio. Uma \textit{support vector machine} (SVM) foi usada como classificador. De  acordo com os experimentos, a \textit{taxa de verdadeiros positivos} é de 98,75\% e a \textit{taxa de verdadeiros negativos} é de 98,75\%.
\\
\par Em \cite{DiqunYan2019}, é mostrado um método para diferenciar a voz de um locutor verdadeiro da voz gerada por sistemas usando sintetizadores baseados no \textit{modelo oculto de Markov} (HMM). SAS\cite{SAS2019} foi a escolha para a base de dados. Este método usa coeficientes de características logarítmicos extraídos de \textit{wavelets} que são apresentados a um classificador SVM. Os resultados obtidos implicaram, em média, mais de 99\% de acurácia.
\\
\par Usando uma decomposição por espalhamento baseada em \textit{wavelets} e convertendo o resultado em coeficientes cepstrais (SCCs), o artigo \cite{7802552} contém a descrição de um vetor de características que é avaliado por modelos de mistura Gaussiana (GMM). SAS e ASVspoof 2015 \cite{ASVspoof2015} foram as bases de dados escolhidas para testes. Em relação aos resultados, foram usadas a \textit{taxa de falsos verdadeiros} (FAR) que representa a taxa de ocorrências falsas classificadas como verdadeiras e a \textit{taxa de falsos falsos} (FRR) que é a taxa de ocorrências verdadeiras classificadas como falsas. A \textit{taxa de erros iguais} (ERR), que é o valor de $\dfrac{FAR}{FRR}$, foi 0,18.
\\
\par Em \cite{alluri2019replay}, os autores usam o \textit{"zero time windowing"} ou janelamento de tempo zero (ZTW), conceito esse que deve ser melhor entendido durante a confecção da dissertação, para, em conjunto com a análise cepstral do espectro gerado, fazer a análise do sinal. Os experimentos foram feitos usando-se a base ASVspoof 2017\cite{ASVspoof2017} com um classificador GMM, a taxa geral de ERR dos experimentos foi de 0,1475.
\\
\par Em \cite{8725688}, é citado que existe uma diferença entre as propriedades espectrais da voz original e da voz gravada. São usados coeficientes cepstrais sobre os quais são aplicados uma média e uma normalização de variância para diminuir o impacto dos ruídos na classificação. Uma GMM foi usada como classificador. A base de dados usada é a ASVspoof 2017. Quanto aos resultados, se obteve uma EER geral menor que 0,1.
\\
\par A proposta de \cite{Hanilci2018} foi a de usar sinais residuais de predição linear, para, juntamente com coeficientes cepstrais, criar características que são apresentas a um classificador GMM. Novamente, a base de dados usada foi a ASVspoof 2015 e os resultados em ERR geral foram de 5,249.
\\
\par Para detecção de \textit{voice spoofing} \cite{ISI:000473343500086}, utilizou-se o conceito chamado de ``texturas de voz''. Padrões binários locais (LBP) e seus respectivos histogramas são usados para a construção do vetor de características que foram avaliado por uma SVM. A base de dados usada para testes foi a ASVspoof 2015 e a taxa máxima de acurácia conseguida foi de 0,7167.
\\
\par Uma abordagem que combina análise de sinal de fala usando a \textit{transformada de constante Q} (CQT) com o processamento cepstral é mostrada em \cite{TODISCO2017516}. Essa técnica resulta no que se chama \textit{coeficientes cepstrais de constante Q}(CQCCs). Segundo os autores, a vantagem desses coeficientes é a resolução espectro-temporal variável. As base de dados usadas foram a RedDots \cite{redDots}, ASVspoof 2015 e AVSpoof. Foram usados três classificadores:
\begin{itemize}
\item DA-IICT: uma fusão de dois classificadores GMM, sendo que um deles usa \textit{coeficientes cepstrais de frequência MEL}(MFCC) e o outro usa características CFCC-IF \cite{Patel2015};
\item STC \cite{7472724};
\item SJTU \cite{korshunov2016overview}.
\end{itemize}			
Na seção de experimentos, são feitos testes para cada uma das bases com os seguintes resultados: ASVspoof 2015 $\rightarrow$ EER geral de 0.026; AVspoof $\rightarrow$ EER geral de 0; RedDots $\rightarrow$ EER geral de 0,185.
\\
\par No artigo \cite{ISI:000490497200068}, propõe-se uma aproximação usando reverberação e as partes não vozeadas da fala. Três GMMs foram definidos para a classificação, sendo que esses classificadores ``votam'' se uma ocorrência é ou não verdadeira, ganhado sempre a classificação que obtiver mais votos. A base de dado utilizada foi a ASVSpoof 2017. O sistema de avaliação de acurácia escolhido, novamente, foi o ERR e esta alcançou um valor de 2,99.
\\
\par A principal ideia contida no artigo \cite{ISI:000465363900136} é capturar a amplitude instantânea vinda de flutuações de energia. Segundo o artigo, as modulações de amplitude são mais suscetíveis ao ruído inserido no sinal original por uma fonte reprodutora. O estudo usa a base de dados ASVSpoof 2017 e GMM como classificador. Os resultados apresentados chegaram a uma EER de 0.0019.
\\
\par No trabalho \cite{ISI:000465363900139} foram usadas as diferenças entre bandas de frequências específicas para diferenciar um sinal legítimo de um usado em ataques de falsificação. No trabalho, é proposta a \textit{predição linear em domínio de frequência}(FDLP) juntamente com GMMs para classificação dos dados presentes na base ASVspoof  2017. Os resultados apresentados chegaram a uma EER de 0.0803.
\\
\par Em \cite{Suthokumar2018} se propõe duas novas características que visam interpretar as componentes estáticas e dinâmicas do sinal. Essas características complementam as características de tempo restrito no espectro. São elas a \textit{modulation  spectral  centroid  frequency} e a \textit{long term spectral average}. O sistema usa como classificador um GMM juntamente com a base dados ASVSpoof 2017. Os resultados chegaram a um valor de EER de 0,0654.
\\
\par Considerando o envelopamento das amplitudes e das frequências instantâneas em cada banda estreita filtrada, o artigo \cite{ISI:000458728700054} contêm uma discussão sobre como diferenciar um sinal de voz legítimo de um falso. A base de dados usada foi a ASVSpoof 2015. Um GMM foi usado como classificador e, em relação a acurácia, o método chegou a ter um EER de 0,045.
\\
\par No trabalho \cite{ISI:000392503100008}, foi proposto o uso do \textit{gammatone frequency cepstral coefficients}(MGFCC). O gammatone é o produto de uma distribuição gamma com um sinal senoide e é usado na construção de filtros auditivos que, neste caso, são usados para extrair características do sinal de voz. A base de dados usada foi a ASVspoof 2015. O classificador usado foi um GMM e o EER chegou a 0,02556.
\\
\par Segundo \cite{8396208}, \textit{Hashing} sensível a locus(LSH) é frequentemente usado como um classificador para problemas relacionados a \textit{big data}. No respectivo trabalho, é proposto uma junção de MFCC e LSH a fim de se reconhecer o locutor. O MFCC é extraído dos arquivos de sinal para posterior aplicação do LSH gerando assim uma tabela \textit{hash} e esses valores de \textit{hash} são então comparados identificando assim o locutor ou locutora. Nos testes realizados houve uma acurácia de 92,66\%. A base de dados usada foi a TIMIT 2018 \cite{TIMIT2018}. 
\subsection{Contextualização}
\par No trabalho proposto para dissertação de mestrado do autor, a intenção é encontrar um conjunto de características que demonstrem ser as mais disjuntas possíveis para fins de separação entre as classes ``locutor autêntico'' e ``ataque de \textit{voice spoofing}''. As caraterísticas serão obtidas no domínio \textit{wavelet}, devido a sua boa resolução em relação às dimensões de tempo e frequência. Essas características serão avaliadas usando a análise paraconsistente de acordo com o trabalho \cite{8588433}, recentemente publicado, e então comparados com trabalhos do estado-da-arte, tais como os citados nesta seção. 
\end{myenv}

\chapter{Cronograma para conclusão do curso de mestrado}
\begin{myenv}{1.5}
\par Até a presente data, foram realizados os primeiros levantamentos para construção da base de dados com as vozes que serão objeto de pesquisa. Essa base conta com um total de 21 gravações originais e outras 21 de \textit{playback} gravadas de pessoas com variados gêneros e idades pronunciando os dígitos de zero a nove em Inglês. A ideia é que essa base possa crescer e abarcar boa parte dos tipos de vozes existentes na região geográfica próxima ao campus da UNESP de São José do Rio Preto. Também foram revisados e estudados conceitos envolvendo filtros passa-baixas e passa-altas, \textit{wavelets} e metodologias de criação de vetores de características usando os intervalos espectrais pré-definidos pelas bandas MEL e BARK, as quais serão implementadas no domínio \textit{wavelet}. Foi também melhorada a biblioteca criada e fornecida pelo orientador deste trabalho, a qual facilita a manipulação dos arquivos de áudio no formato \textit{wave}. Do mesmo modo, código-fonte para análise paraconsistente de características foi também desenvolvido e complementado com a devida documentação, a qual foi norteada por instruções recebidas do orientador.
\\
\par Quanto aos passos futuros, a tabela \ref{Cronograma} contêm o cronograma previsto. Acredita-se, diante do exposto, que o andamento dos trabalhos estejam em um nível coerente e dentro das expectativas.

			\newcommand\ytl[2]{
	\parbox[b]{10em}{
		\hfill{\color{black}\bfseries\sffamily #1}~$\cdots\cdots$~
	}
	\makebox[0pt][c]{
		$\bullet$
	}\vrule\quad \parbox[c]{4.5cm}{
		\vspace{7pt}\color{black}\raggedright\sffamily #2.\\[7pt]
	}\\
	[-3pt]
}

\begin{table}
	\centering
	\ytl{09/03 -- 31/03}{Início dos trabalhos: Coleta de dados para constituição da base de dados e estudo da base bibliográfica. Escrita da dissertação}
	\ytl{01/04 -- 15/04}{Coleta de dados para constituição da base de dados e estudo da base bibliográfica. Início dos experimentos. Escrita da dissertação}
	\ytl{16/04 -- 16/04}{Reunião de validação com o orientador}
	\ytl{17/04 -- 29/04}{Experimentos e escrita da dissertação}
	\ytl{30/04 -- 30/04}{Reunião de validação com o orientador}
	\ytl{01/05 -- 20/05}{Experimentos e escrita da dissertação}
	\ytl{21/05 -- 28/05}{Reunião de validação com o orientador}
	\ytl{29/05 -- 10/06}{Experimentos e escrita da dissertação}
	\ytl{11/06 -- 11/06}{Reunião de validação com o orientador}
	\ytl{12/06 -- 17/06}{Experimentos e escrita da dissertação}
	\ytl{18/06 -- 18/06}{Reunião de validação com o orientador}
	\ytl{19/06 -- 24/06}{Experimentos e escrita da dissertação}
	\ytl{25/06 -- 25/06}{Reunião de validação com o orientador}
	\ytl{26/06 -- 01/07}{Experimentos e escrita da dissertação}
	\ytl{02/07 -- 02/07}{Reunião de validação com o orientador}
	\ytl{03/07}{Entrega da dissertação}
	\caption{cronograma}
	\label{Cronograma}
\end{table}

	\end{myenv}

	\bibliography{bibliography.bib}

	\bibliographystyle{alpha}

\end{document}